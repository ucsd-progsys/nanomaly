\section{Proofs for Section~\ref{sec:searching-witness}}
\label{sec:proofs}

% \begin{proof}[Proof of Lemma~\ref{lem:force-inst}]
%   By case analysis on the evaluation rules.
%   %
%   If $\ptype{\trace}{f} \neq \ptype{\trace'}{f}$ then,
%   % one of the holes in $f$'s
%   % argument must have been instantiated with a concrete value at the last step.
%   by the definition of $\ptype{\trace}{f}$, $\tsu \neq \tsu'$, as \thole
%   does not change.
%   %
%   % An examination of the rules shows that only place this happens is
%   % in the second case of \forcesym.
%   An examination of the rules shows that only \forcesym can update \tsu.


%   Furthermore, an examination of \forcesym immediately shows that in the
%   cases where \forcesym returns \stuck, \tsu\ is unmodified. Thus only
%   \emph{successful} calls to \forcesym can change \tsu\ and, by
%   extension, \ptype{\trace}{f}.

%   % - narrow(leaf[t_1], t_2, \sigma, \theta)
%   %   - t_1 != \alpha because E-Leaf always creates fresh \alpha
%   %   - what aboue t_2? could mention \alpha in E-Node-*..

%   % By case analysis on the evaluation rules. As above we are only
%   % concerned with the sucessful steps, and can thus ignore the
%   % \rulename{E-*-Bad} rules.

%   % In each case we will show that only
%   % the calls to \forcesym can change the partial input type of $f$.

%   % \begin{description}
%   % % \item[Case \reholegood] \hastype{\ehole}{\thole}, which is
%   % %   alpha-equivalent to \typeof{\vhole{\thole}}, thus this step does not
%   % %   change the partial type.
%   % \item[Case \replusgood] Since narrowing succeeded on $v_1$ and $v_2$,
%   %   they must have both either been ints or unconstrained holes.
%   %   The only way the partial input type could have changed is if one or the
%   %   other was a hole, and was instantiated by \forcesym.
%   % \item[Case \rulename{E-If-Good-\{1,2\}}] Since narrowing succeeded on
%   %   $v$, it must have either been a bool or an unconstrained hole.
%   %   The only way the partial input type could have changed is if $v$ was a
%   %   hole and was instantiated by \forcesym.
%   % \item[Case \reappgood] Since narrowing succeeded on $v$, it must have
%   %   either been a lambda or an unconstrained hole. The only way the
%   %   partial input type could have changed is if $v$ was a hole and was
%   %   instantiated by \forcesym.
%   % \item[Case \releafgood] This rule does not call \forcesym, thus it
%   %   cannot change the partial input type, and cannot have been applied.
%   % \item[Case \renodegood] In this case the partial input type could have
%   %   changed if any of $v_1$, $v_2$, or $v_3$ were unconstrained holes.
%   %   Since both \forcesym calls succeeded \emph{and} the partial input
%   %   type changed

%   %   Since narrowing succeeded on $v_2$ and $v_3$,
%   %   they must have both been trees or unconstrained holes

%   %   \ES{come back to this, not entirely clear it
%   %     holds as we might learn the element type from one of the subtrees}
%   % \item[Case \rulename{E-Case-Good\{1,2\}}] Since narrowing succeeded on
%   %   $v$, it must have either been a tree or an unconstrained hole.
%   %   The only way the partial input type could have changed is if $v$ was a
%   %   hole and was instantiated by \forcesym.
%   % \end{description}
% \end{proof}


% \begin{proof}[Proof of Lemma~\ref{lem:refine-partial}]
%   By case analysis on the evaluation rules.

%   We can immediately discharge the $\rulename{E-*-Bad}$ rules as they
%   result in the \stuck state, and by Lemma~\ref{lem:force-inst} we know
%   that these rules cannot change \ptype{\trace}{f} at all.

%   % and
%   % an inspection of \forcesym shows that when \forcesym returns \stuck,
%   % it does not modify \vsu or \tsu. Furthermore, \forcesym is the only
%   % procedure that can modify the substitutions.

%   % For the remainder of the rules we must
%   % consider how they might affect both the input (by instantiating holes)
%   % and output components of the partial type.

%   % We will call a type hole \thole \emph{unconstrained} if applying the
%   % current type substitution \subst{\tsu}{\thole} produces a type hole.

%   \begin{description}
%   \item[Case \replusgood] Both $v_1$ and $v_2$ successfully narrowed to
%     \tint, so they must have either been ints or unconstrained
%     holes. Thus partial input type compatibility is preserved.

%     % \hastype{\eplus{v_1}{v_2}}{\thole} and
%     % \hastype{n}{\tint}, so partial type compatibility of the output is
%     % preserved.
%   \item[Case \rulename{E-If-Good-\{1,2\}}] $v$ successfully narrowed to
%     \tbool, so it must have either been a bool or an unconstrained hole.
%     Thus partial input type compatibility is preserved.

%     % \hastype{\eif{v}{e_1}{e_2}}{\thole}, which is compatible with any
%     % type $e_1$ or $e_2$ might have, so partial type compatibility of the
%     % output is also preserved.
%   \item[Case \reappgood] $v_1$ successfully narrowed to \tfun, so it
%     must have either been a lambda or an unconstrained hole, so partial
%     input type compatibility is preserved.

%     % \hastype{\eapp{v_1}{v_2}}{\thole}, which is compatible with any type
%     % $e$ might have, so partial type compatiblity of the output is
%     % preserved.
%   \item[Case \releafgood] This rule does not call \forcesym, so this
%     step cannot affect partial input type compatibility.

%     % \eleaf steps to \vleaf{\thole} with a fresh \thole,
%     % but \hastype{\eleaf}{\ttree{\thole}}, so partial type compatibility
%     % of the output is preserved.
%   \item[Case \renodegood]
%     % $v_1$ successfully narrowed to $t$, so it must
%     % have already been a $t$ or an unconstrained hole. Likewise,
%     $v_2$ and $v_3$ successfully narrowed to \ttree{t}, so they must have
%     already been \ttree{t}'s or unconstrained holes. Thus partial input type
%     compatibility is preserved.
%     %
%     $v_1$ is not narrowed, but if it \vhole{\thole} we may constrain
%     \thole by narrowing $v_2$ or $v_3$. However, partial input type
%     compatibility must still be preserved as the calls to \forcesym will
%     only succeed if \thole is either completely unconstrained (in which
%     case compatibility is trivial), or if it were constrained to a type
%     that is compatible with the types of $v_2$ and $v_3$.

%     % so if it was a hole it would not be
%     % instantiated here, thus it cannot affect partial input type compatibility.

%     % \hastype{\enode{v_1}{v_2}{v_3}}{\ttree{\thole}}, with a fresh \thole
%     % and\\ \hastype{\vnode{t}{v_1}{v_2}{v_3}}{\ttree{t}}. But
%     % \tcompat{\thole}{\ttree{t}} because we can just map \thole to
%     % \ttree{t} (as \thole is fresh), so partial type compatibility of the
%     % output is preserved.
%   \item[Case \rulename{E-Case-Good-\{1,2\}}] $v$ successfully narrowed to
%     \ttree{\thole} so it must have either been a tree or an
%     unconstrained hole, so partial input type compatibility is
%     preserved.

%     % \hastype{\ecase{v}{e_1}{x_1}{x_2}{x_3}{e_2}}{\thole}, which is
%     % compatible with any type $e_1$ and $e_2$ might have, so partial type
%     % compatibility of the output is preserved.
%   % \item[Case \reholegood] \hastype{\ehole}{\thole}, with a fresh \thole,
%   %   and \hastype{\vhole{\thole}}{\thole}, but a fresh hole is compatible
%   %   with anything, so partial input type compatibility is preserved.
%   \end{description}
% \end{proof}

% \begin{proof}[Proof of Lemma~\ref{lem:refine-partial}]
%   By case analysis on the evaluation rules. We can immediately discharge
%   the $\rulename{E-*-Bad}$ rules as they result in the \stuck state.
%   \begin{description}
%   \item[Case \replusgood] \hastype{\eplus{v_1}{v_2}}{\thole} and
%     \hastype{n}{\tint}, so partial type compatibility is preserved.
%   \item[Case \rulename{E-If-Good\{1,2\}}]
%     \hastype{\eif{v}{e_1}{e_2}}{\thole}, which is compatible with any
%     type $e_1$ or $e_2$ might have.
%   \item[Case \reappgood] \hastype{\eapp{v_1}{v_2}}{\thole}, which is
%     compatible with any type $e$ might have.
%   \item[Case \eleaf] \eleaf steps to \vleaf{\thole} with a fresh \thole,
%     but \hastype{\eleaf}{\ttree{\thole}}, so partial type compatibility
%     is preserved.
%   \item[Case \renodegood]
%     \hastype{\enode{v_1}{v_2}{v_3}}{\ttree{\thole}}, with a fresh \thole
%     and \hastype{\vnode{t}{v_1}{v_2}{v_3}}{\ttree{t}}. But
%     \tcompat{\thole}{\ttree{t}} because we can just map \thole to
%     \ttree{t} (as \thole is fresh), so partial type compatibility is
%     preserved.
%   \item[Case \rulename{E-Case-Good\{1,2\}}]
%     \hastype{\ecase{v}{e_1}{x_1}{x_2}{x_3}{e_2}}{\thole}, which is compatible
%     with any type $e_1$ and $e_2$ might have.
%   \item[Case \reholegood] \hastype{\ehole}{\thole}, with a fresh
%     \thole, and \hastype{\vhole{\thole}}{\thole}, but a fresh hole is
%     compatible with anything, so compatibility is preserved.
%   \end{description}
% \end{proof}

% \begin{lem}
% \label{lem:resolve-compat}
% For all traces
% $\trace \defeq \triple{\eapp{f}{\vhole{\thole}}}{\emptysu}{\emptysu},\ldots,\triple{e}{\vsu}{\tsu}$,
% $\vsub{\resolve{\thole}{\tsu}}{\resolve{\ehole}{\vsu}}$.
% \end{lem}
\begin{proof}[Proof of Lemma~\ref{lem:resolve-compat}]
  By induction on $\trace$.
  %
  In the base case $\trace = \triple{\eapp{f}{\vhole{\thole}}}{\emptysu}{\emptysu}$
  and $\thole$ is trivially a refinement of $\vhole{\thole}$.
  %
  In the inductive case, consider the single-step extension of $\trace$,
  $\trace' = \trace,\triple{e'}{\vsu'}{\tsu'}$.
  %
  We show by case analysis on the evaluation rules that if
  $\vsub{\resolve{\thole}{\tsu}}{\resolve{\ehole}{\vsu}}$, then
  $\vsub{\resolve{\thole}{\tsu'}}{\resolve{\ehole}{\vsu'}}$.

  We can immediately discharge all of the \rulename{E-*-Bad} rules
  (except for \renodebadone) as the calls to \forcesym\ return \stuck.
  An examination of \forcesym shows that if \forcesym\ returns \stuck\
  then \vsu\ and \tsu\ are unchanged.
  \begin{description}
  \item[Case \replusgood:]
    %
    We \forcesym\ $v_1$ and $v_2$ to $\tint$, so we must consider
    the $\force{\vhole{\thole}}{t}{\vsu}{\tsu}$ and
    $\force{n}{\tint}{\vsu}{\tsu}$ cases.
    %
    The $\force{n}{\tint}{\vsu}{\tsu}$ case is trivial as it does
    not change \vsu\ or \tsu.
    %
    In the $\force{\vhole{\thole}}{t}{\vsu}{\tsu}$ we will either
    find that $\ehole \in \vsu$ or we will generate a fresh \tint
    and extend \vsu.
    %
    Note that when we extend \vsu\ we also extend \tsu\ due to the
    call to \unifysym, thus in the $\vhole \in \vsu$ we cannot
    actually refine either $\ehole$ or $\thole$ and thus the
    refinement is preserved.
    %
    When we extend \vsu\ with a binding for \ehole, the call to
    \unifysym ensures that we add a compatible binding for
    \thole if one was not already in \tsu, thus the refinement
    relation must continue to hold.
  \item[Case \rulename{E-If-Good\{1,2\}}:]
    %
    Similar to \replusgood.
  \item[Case \reappgood:]
    %
    Similar to \replusgood.
  \item[Case \releafgood:]
    %
    This step cannot change \vsu\ or \tsu\ thus the refinement
    relation continues to hold trivially.
  \item[Case \renodegood:]
    %
    We \forcesym\ $v_2$ and $v_3$ to $\ttree{t}$, so we must consider
    three cases of \forcesym.
    \begin{description}
    \item[\force{\vhole{\thole}}{t}{\vsu}{\tsu}:]
      %
      Similar to \replusgood.
    \item[\force{\vleaf{t_1}}{\ttree{t_2}}{\vsu}{\tsu}:]
      %
      This case may extend \tsu\ but not \vsu, so the refinement
      continues to hold trivially.
    \item[\force{\vnode{t_1}{v_1}{v_2}{v_3}}{\ttree{t_2}}{\vsu}{\tsu}:]
      %
      Same as \vleaf{t_1}.
    \end{description}
  \item[Case \rulename{E-Case-Good\{1,2\}}:]
    %
    Similar to \replusgood.
  \item[Case \repcasegood:]
    %
    Similar to \replusgood.
  \end{description}
\end{proof}


% \begin{proof}[Proof of Lemma~\ref{lem:vsu-ext}]
%   By case analysis on the evaluation rules.
%   %
%   $\thole$ does not change, so if the resolved types differ then
%   $\tsu \neq \tsu'$.
%   %
%   Note that only \forcesym can change \tsu, and it can only do so
%   via \unifysym, which can only extend \tsu.
%   \begin{description}
%   \item[Case \replusgood:]
%     %

%   \end{description}
% \end{proof}
% \begin{lem}
% \label{lem:context-compat}
% For all types $s$ and $t$,
% if $\tcompat{\intctx{s}}{t}$
% then $t = \intctx{s'}$
% such that $\tcompat{s}{s'}$.
% \end{lem}
% \begin{proof} [Proof of Lemma~\ref{lem:context-compat}]
%   % By induction on $T$.

%   % In the base case $T$ must be $\bullet$, which means that $t = s'$
%   % and thus $\tcompat{s}{s'}$ holds trivially.

%   % In the inductive case let $T = \ttree{T'}$. First note that in order
%   % to satisfy the antecedent, $t = \ttree{t'}$. Furthermore, the
%   % inductive hypothesis tells us that if $\tcompat{T'[s]}{t'}$ then
%   % $t' = T'[s']$ such that $\tcompat{s}{s'}$.

%   By induction on $t$.
%   %
%   In the base case $t$ must be one of $\tint$, $\tbool$, $\tfun$, or
%   some $\thole$, and $T$ must be $\bullet$. This means that $s = s'$
%   which implies that $\tcompat{s}{s'}$.
%   %
%   In the inductive case we must consider
%   $t = \ttree{t'}$,
%   $t = \tprod{t'}{t''}$, and
%   $t = \tprod{t''}{t'}$,
%   and show that
%   % let $t = \ttree{t'}$, we show that
%   if $\tcompat{T'[s]}{t'}$ implies that $t' = T'[s']$ such that
%      $\tcompat{s}{s'}$,
%   then $\tcompat{T[s]}{t}$ implies that $t = T[s']$ such that
%        $\tcompat{s}{s'}$.
%   %
%   \begin{description}
%   \item[Case $t = \ttree{t'}$:]
%     %
%     If $\tcompat{T[s]}{\ttree{t'}}$ then either $T = \ttree{T'}$, in
%     which case the inductive hypothesis applies, or $T = \bullet$, in
%     which case $s = s'$ and the consequent holds trivially.
%   \item[Case $t = \tprod{t'}{t''}$:]
%     %
%     If $\tcompat{T[s]}{\tprod{t'}{t''}}$
%     then either $T = \tprod{T'}{s''}$ such that $\tcompat{t''}{s''}$, in
%     which case the inductive hypothesis applies, or $T = \bullet$, in
%     which case $s = s'$ and the consequent holds trivially.
%   \item[Case $t = \tprod{t''}{t'}$:] Symmetric to $t = \tprod{t'}{t''}$.
%   \end{description}
% \end{proof}

\begin{proof}[Proof of Lemma~\ref{lem:k-stuck}]
  We can construct $v$ from $\trace$ as follows.
  %
  Let
  $$
  \trace_i = \triple{\eapp{f}{\vhole{\thole}}}{\emptysu}{\emptysu},
             \ldots,
             \triple{e_{i-1}}{\vsu_{i-1}}{\tsu_{i-1}},
             \triple{e_{i}}{\vsu_{i}}{\tsu_{i}}
  $$
  be the shortest prefix of $\trace$ such that
  $\tincompat{\ptype{\trace_i}{f}}{t}$.
  % $\tincompat{\resolve{\thole}{\tsu_{i}}}{t}$.
  %
  We will show that $\ptype{\trace_{i-1}}{f}$ % $\resolve{\thole}{\tsu_{i-1}}$
  must contain some other hole $\thole'$ that is
  instantiated at step $i$.
  %
  Furthermore, $\thole'$ is instantiated in such a way that
  $\tincompat{\ptype{\trace_i}{f}}{t}$.
  %
  Finally, we will show that if we had instantiated $\thole'$ such that
  $\tcompat{\ptype{\trace_i}{f}}{t}$,
  % $\tcompat{\resolve{\thole}{\tsu_{i}}}{t}$,
  the current step would have gotten $\stuck$.

  % By Lemma~\ref{lem:fixme} we know that
  % $\vcompat{\resolve{\vhole{\thole}}{\vsu}}{\resolve{\thole}{\tsu}}$.
  %
  Since $\tsu_{i-1}$ and $\tsu_{i}$ differ only in $\thole'$ but the resolved
  types differ, we have
  $\thole' \in \ptype{\trace_{i-1}}{f}$
  and
  $\ptype{\trace_i}{f} = \ptype{\trace_{i-1}}{f}\sub{\thole'}{t'}$.
  %
  % We prove\includeTechReport{, in Lemma~\ref{lem:context-compat},}
  % that for all types $s$ and $t$, if.
  Let $s$ be a
  concrete type such that $\ptype{\trace_{i-1}}{f}\sub{\thole'}{s} = t$.
  %
  We show by case analysis on the evaluation rules that
  %
  $$\step{e_{i-1}}{\vsu_{i-1}}{\tsu_{i-1}[\thole' \mapsto s]}{\stuck}{\vsu}{\tsu}$$
  %
    \begin{description}
    \item[Case \replusgood:]
      %
      Here we \forcesym $v_1$ and $v_2$ to $\tint$, so the first case of
      \forcesym must apply ($\force{n}{\tint}{\vsu}{\tsu}$ cannot apply
      as it does not change $\tsu$).
      %
      In particular, since we extended $\tsu_{i-1}$ with
      $[\thole' \mapsto t']$ we know that $\thole' = \thole$ and
      $t' = \tint$.
      %
      Let $s$ be any concrete type that is incompatible with $\tint$
      and $\tsu_s = \tsu_{i-1}[\thole \mapsto s]$,
      $\force{\vhole{\thole}}{\tint}{\vsu_{i-1}}{\tsu_s]} = \triple{\stuck}{\vsu_{i-1}}{\tsu_s}$.
    \item[Case \rulename{E-Plus-Bad\{1,2\}}:]
      %
      These cases cannot apply as \forcesym does not update \tsu\ when
      it returns \stuck.
    \item[Case \rulename{E-If-Good\{1,2\}}:]
      %
      Similar to \replusgood.
      % Here we \forcesym $v$ to $\tbool$, so the first case of \forcesym
      % must apply ($\force{b}{\tbool}{\vsu}{\tsu}$ cannot apply as it
      % does not change $\vsu$).
      % %
      % In particular, we must have extended $\vsu$ with
      % $[\ehole' \mapsto v']$ where $v'$ is a $\tbool$.
      % %
      % Let $u$ be any concrete value that is incompatible with $\tbool$,
      % $\force{u}{\tbool}{\vsu}{\tsu} = \triple{\stuck}{\vsu}{\tsu}$.
    \item[Case \reifbad:]
      %
      This case cannot apply as \forcesym does not update \tsu\ when
      it returns \stuck.
    \item[Case \reappgood:]
      %
      Similar to \replusgood.
      % Here we \forcesym $v$ to $\tfun$, so the first case of \forcesym
      % must apply ($\force{\efun{x}{e}}{\tfun}{\vsu}{\tsu}$ cannot apply as it
      % does not change $\vsu$).
      % %
      % In particular, we must have extended $\vsu$ with
      % $[\ehole' \mapsto v']$ where $v'$ is a $\tfun$.
      % %
      % Let $u$ be any concrete value that is incompatible with $\tfun$,
      % $\force{u}{\tfun}{\vsu}{\tsu} = \triple{\stuck}{\vsu}{\tsu}$.
    \item[Case \reappbad:]
      %
      This case cannot apply as \forcesym does not update \tsu\ when
      it returns \stuck.
    \item[Case \releafgood:]
      %
      This case cannot apply as it does not update \tsu.
    \item[Case \renodegood:]
      %
      Here we \forcesym $v_2$ and $v_3$ to $\ttree{t}$,
      so we must consider three cases of \forcesym.
      \begin{description}
      \item[\force{\vhole{\thole}}{t}{\vsu}{\tsu}:]
        %
        Similar to \replusgood.
      \item[\force{\vleaf{t_1}}{\ttree{t_2}}{\vsu}{\tsu}:]
        %
        For this case to extend \tsu\ with $[\thole' \mapsto t']$,
        either $t_1$ or $t_2$ must contain $\thole'$.
        %
        Let $s$ be any concrete type that is incompatible with $t'$
        and $\tsu_s = \tsu_{i-1}[\thole \mapsto s]$,
        $\force{\vhole{\thole}}{\tint}{\vsu_{i-1}}{\tsu_s]} = \triple{\stuck}{\vsu_{i-1}}{\tsu_s}$.
      \item[\force{\vnode{t_1}{v_1}{v_2}{v_3}}{\ttree{t_2}}{\vsu}{\tsu}:]
        %
        Same as \vleaf{t_1}.
      \end{description}
      % so the first
      % case of \forcesym must apply
      % (neither\\ $\force{\vleaf{\ttree{t_1}}}{\ttree{t_2}}{\vsu}{\tsu}$
      %  nor\\ $\force{\vnode{\ttree{t_1}}{v_1}{v_2}{v_3}}{\ttree{t_2}}{\vsu}{\tsu}$
      %  can apply as they do not change $\vsu$).
      % %
      % In particular, we must have extended $\vsu$ with
      % $[\ehole' \mapsto v']$ where $v'$ is a $\ttree{t}$.
      % %
      % Let $u$ be any concrete value that is incompatible with $\ttree{t}$,
      % $\force{u}{\ttree{t}}{\vsu}{\tsu} = \triple{\stuck}{\vsu}{\tsu}$.
      % %
      % \ES{this seems too easy...}
    \item[Case \renodebadone:]
      %
      This case cannot apply as \forcesym does not update \tsu\ whe
      it returns \stuck.
    \item[Case \renodebadtwo:]
      %
      Similar to \renodegood.
      % In this case $\force{v_2}{\ttree{t}}{\vsu_1}{\tsu_1}$ can update
      % \vsu, but $\force{v_3}{\ttree{t}}{\vsu_2}{\tsu_2}$ can not, as the
      % latter returns \stuck.
      % %
      % For the former case we can apply the same reasoning as for
      % \renodegood to derive a value $u$ that will cause stuckness.
    \item[Case \rulename{E-Case-Good\{1,2\}}:]
      %
      Here we \forcesym $v$ to $\ttree{\thole}$,
      so we must consider three cases of \forcesym.
      \begin{description}
      \item[\force{\vhole{\thole}}{t}{\vsu}{\tsu}:]
        %
        Similar to \replusgood.
      \item[\force{\vleaf{t_1}}{\ttree{t_2}}{\vsu}{\tsu}:]
        %
        This case cannot extend \tsu\ with $[\thole' \mapsto t']$ as we
        use a fresh $\thole$, which cannot be referenced by
        $\ptype{\trace_{i-1}}{f}$, in the call to \forcesym, and thus it
        cannot apply.
      \item[\force{\vnode{t_1}{v_1}{v_2}{v_3}}{\ttree{t_2}}{\vsu}{\tsu}:]
        %
        Same as \vleaf{t_1}.
      \end{description}
      % Here we \forcesym $v$ to $\ttree{\thole}$, so the first case of \forcesym
      % must apply
      % (neither\\ $\force{\vleaf{\ttree{t_1}}}{\ttree{t_2}}{\vsu}{\tsu}$
      %  nor\\ $\force{\vnode{\ttree{t_1}}{v_1}{v_2}{v_3}}{\ttree{t_2}}{\vsu}{\tsu}$
      %  can apply as they do not change $\vsu$).
      % %
      % In particular, we must have extended $\vsu$ with
      % $[\ehole' \mapsto v']$ where $v'$ is a $\ttree{\thole}$.
      % %
      % Let $u$ be any concrete value that is incompatible with $\ttree{\thole}$,
      % $\force{u}{\ttree{\thole}}{\vsu}{\tsu} = \triple{\stuck}{\vsu}{\tsu}$.
    \item[Case \recasebad:]
      %
      This case cannot apply as \forcesym does not update \tsu\ whe
      it returns \stuck.
    \item[Case \repcasegood]
      %
      Here we \forcesym $v$ to $\tprod{\thole_1}{\thole_2}$,
      so we must consider two cases of \forcesym.
      \begin{description}
      \item[\force{\vhole{\thole}}{t}{\vsu}{\tsu}:]
        %
        Similar to \replusgood.
      \item[\force{\epair{v_1}{v_2}}{\tprod{t_1}{t_2}}{\vsu}{\tsu}:]
        %
        This case cannot extend \tsu\ with $[\thole' \mapsto t']$ as we
        use a fresh $\thole_1$ and $\thole_2$, which cannot be
        referenced by $\ptype{\trace_{i-1}}{f}$, in the call to
        \forcesym, and thus it cannot apply.
      \end{description}
    \item[Case \repcasebad:]
      %
      This case cannot apply as \forcesym does not update \tsu\ whe
      it returns \stuck.
    \end{description}
  %
  Finally, by Lemma~\ref{lem:resolve-compat} we know that
  % $\vsub{\resolve{\thole}{\tsu_{i-1}}}{\resolve{\ehole}{\vsu_{i-1}}}$,
  $\vsub{\ptype{\trace_{i-1}}{f}}{\resolve{\ehole}{\vsu_{i-1}}}$
  and thus $\thole' \in \resolve{\vhole{\thole}}{\vsu_{i-1}}$.
  %thus $\vsub{\intctx{\thole'}}{\resolve{\ehole}{\vsu_{i-1}}}$.
  %
  Let $u = \gen{s}{\tsu}$
  and $v = \resolve{\ehole}{\vsu_{i-1}}\sub{\ehole'[\thole']}{u}\sub{\thole'}{s}$,
  $\steps{\eapp{f}{v}}{\emptysu}{\emptysu}{\stuck}{\vsu}{\tsu}$ in $i$ steps.
  %
  % \begin{description}
  % \item [Case \tincompat{t}{s_{\trace'}}:]
  %   The inductive hypothesis applies.
  % \item [Case $\tcompat{t}{s_{\trace'}}$ but $\tincompat{t}{s_{\trace}}$:]
  %   Since $\tcompat{t}{s_{\trace'}}$ but $\tincompat{t}{s_{\trace}}$ we know
  %   that $s_{\trace'} \neq s_{\trace}$, and
  %   by Lemma~\ref{lem:force-inst} we know that we must have
  %   successfully called \forcesym at step $k$.
  %   %
  %   Lemma~\ref{lem:refine-partial} tells us
  %   $\tcompat{s_{\trace'}}{s_{\trace}}$, which means we must have
  %   specifically narrowed $s_{\trace'}$ to a type incompatible with $t$.
  %   %
  %   We show by case analysis of the evaluation rules that narrowing to
  %   $t$ instead cannot succeed.

  %   Again we can immediately discharge the \rulename{E-*-Bad}
  %   rules as they get stuck, proving the consequent.

  %   In the \rulename{E-*-Good} rules we must consider where a hole might
  %   appear in the expression, and what would happen if we replaced the
  %   hole with the value $w$.
  %   %
  %   \begin{description}
  %   % \item[Case \reholegood:] \thole is compatible with any type, which
  %   %   contradicts the premise that \tincompat{t}{s_{\trace}}.
  %   \item[Case \replusgood:] The hole may appear in either $v_1$ or
  %     $v_2$. We \forcesym $v_1$ and $v_2$ to have type \tint, but we
  %     assumed that $w$ has a type that is incompatible with \tint, thus
  %     if we replace the hole with $w$, \forcesym must fail on either
  %     $v_1$ or $v_2$.
  %   \item[Case \rulename{E-If-Good-\{1,2\}}:]

  %     By Lemma~\ref{lem:new-lem} we know that $\vhole{\thole} \in v$ and
  %     $[\thole \mapsto t] \in \theta'$. Thus we must have taken the first
  %     case of \forcesym, so $v = \vhole{\thole}$.

  %     % By Lemma~\ref{lem:force-inst} we know that this step must have
  %     % called \forcesym on $\vhole{\thole}$, narrowing it to have type
  %     % \tbool, thus $\ptype{\trace}{f} = \tbool$. But we assumed
  %     % $\vincompat{w}{\tbool}$, thus \forcesym cannot succeed if we
  %     % replace $\vhole{\thole}$ with $w$.
  %     % Here we \forcesym $v$ to have type
  %     % \tbool, but we assumed that $v$ has a type that is incompatible with
  %     % \tbool, thus \forcesym must fail.
  %   \item[Case \reappgood:] Here we \forcesym $v$ to have type
  %     \tfun, but we assumed that $v$ has a type that is incompatible with
  %     \tfun, thus \forcesym must fail.
  %   \item[Case \releafgood:] This rule does not call \forcesym, so by
  %     Lemma~\ref{lem:force-inst} it cannot have applied.
  %   \item[Case \renodegood:] There are two cases we must consider here,
  %     the sub-trees and the value at the node.

  %     Consider first the sub-trees. We \forcesym $v_2$ and $v_3$ to have
  %     type \ttree{t}, but we assumed $v$ has some type incompatible with
  %     \ttree{t}, so one of the \forcesym calls must fail.

  %     Next consider the value $v_1$. Even though we do not \forcesym
  %     $v_1$ directly, we seed the \forcesym calls with $\typeof{v_1}$,
  %     which will then fail if $\typeof{v_1}$ is incompatible with the
  %     types of the sub-trees, which our assumption tells us must be the
  %     case.
  %   \item[Case \rulename{E-Case-Good-\{1,2\}}:] Here we \forcesym $v$ to
  %     have type \ttree{\thole}, but we assumed that $v$ has a type that
  %     is incompatible with \ttree{\thole}, thus \forcesym must fail.
  %   \end{description}
  % \end{description}
\end{proof}

% \begin{proof}[Proof of Theorem~\ref{thm:soundness}]
% Suppose $\trace$ witnesses that $f$ gets stuck,
% and let $t = \ptype{\trace}{f}$.
% We show that \emph{all} types $s$ have stuck-inducing
% values by splitting cases on whether the type is
% compatible with $t$. %the partial type upto $\trace$.
% %
% \begin{description}
% \item [Case \tcompat{s}{t}:]
%   Let $\trace = \triple{\eapp{f}{\vhole{\thole}}}{\emptysu}{\emptysu},\ldots,\triple{\stuck}{\vsu}{\tsu}$.
%   %
%   The value $v = \resolve{\ehole}{\vsu}$ demonstrates that
%   $\eapp{f}{v}$ gets stuck.
% \item [Case \tincompat{s}{t}:] By Lemma~\ref{lem:k-stuck}, we can derive
%   a $v$ from $\trace$ such that \hastype{v}{t} and $\eapp{f}{v}$ gets
%   stuck.
%   % \ES{do we need to say anythign else?}
% \end{description}
% \end{proof}


%%% Local Variables:
%%% mode: latex
%%% TeX-master: "main"
%%% End:
