\begin{abstract}
  %
  \emph{Static} type errors are a common stumbling block
  for newcomers to typed functional languages.
  %
  We present a \emph{dynamic} approach to explaining type
  errors by generating counterexample witness inputs that
  illustrate \emph{how} an ill-typed program goes wrong.
  %
  First, given an ill-typed function, we first symbolically
  execute the body to dynamically synthesize witness values
  that can make the program go wrong.
  We prove that our procedure synthesizes
  \emph{general witnesses} in that if a witness is
  found, then \emph{for all} input types, there exist
  inhabitants that can make the function go wrong.
  %
  Second, we show how to extend the above procedure to
  produce a \emph{reduction graph} that can be used to
  interactively visualize and debug witness executions.
  %
  Third, we evaluate our approach on two data sets
  comprising over 4,500 ill-typed student programs.
  Our technique is able to generate witnesses for
  88\% of the programs, and our reduction graph
  is yields small counterexamples for 90\% of the witnesses.
\end{abstract}
