%
% LaTeX template for prepartion of submissions to PLDI'15
%
% Requires temporary version of sigplanconf style file provided on
% PLDI'15 web site.
%
\documentclass[blind,preprint,nocopyrightspace,10pt,squareay,clearpagebib,explicitsize]{sigplanconf-pldi15}

%
% the following standard packages may be helpful, but are not required
%
\usepackage{amsmath}
\usepackage{amssymb}
% \usepackage{SIunits}            % typset units correctly
\usepackage{courier}            % standard fixed width font
\usepackage[scaled]{helvet} % see www.ctan.org/get/macros/latex/required/psnfss/psnfss2e.pdf
\usepackage{url}                  % format URLs
\usepackage{listings}          % format code
\usepackage{enumitem}      % adjust spacing in enums
\usepackage[colorlinks=true,allcolors=blue,breaklinks,draft=false]{hyperref}   % hyperlinks, including DOIs and URLs in bibliography
% known bug: http://tex.stackexchange.com/questions/1522/pdfendlink-ended-up-in-different-nesting-level-than-pdfstartlink
\newcommand{\doi}[1]{doi:~\href{http://dx.doi.org/#1}{\Hurl{#1}}}   % print a hyperlinked DOI
\usepackage{xspace}
\usepackage{commands}
\usepackage{graphicx}
\usepackage{pdfpages}
%\usepackage[doi]{natbib}

\usepackage{amsthm}
% \newtheorem{lemma}{Lemma}
% \newtheorem{corollary}{Corollary}
% \newtheorem{theorem}{Theorem}
% \newtheorem{definition}{Definition}
\theoremstyle{plain}% default
\newtheorem{thm}{Theorem} % [section]
\newtheorem{lem}[thm]{Lemma}
\newtheorem{prop}[thm]{Proposition}
\newtheorem*{cor}{Corollary}
\theoremstyle{definition}
\newtheorem{defn}{Definition} % [section]

\usepackage{pgfplots}
\usepackage{pgfkeys}
\pgfplotsset{compat=1.12}

\usepackage[inference]{semantic}

\input{haskellListings}



\begin{document}

%
% any author declaration will be ignored  when using 'plid' option (for double blind review)
%

\title{NanoMaLy}

\authorinfo{Eric L. Seidel\and Ranjit Jhala}
           {UC San Diego}
           {{eseidel,rjhala}@cs.ucsd.edu}

\authorinfo{Westley Weimer}
           {University of Virginia}
           {weimer@viginia.edu}

\maketitle
\begin{abstract}
  Type errors are a common stumbling block for newcomers to typed
  functional languages.
  % 
  Compilers prevent standard debugging techniques by refusing to run
  ill-typed programs.
  %
  We demonstrate how to expose static type errors for the runtime
  crashes they hide, by searching for a witness that will trigger the
  error.
  %
  Our search procedure collects an execution trace for which we have
  developed an interactive visualization to demonstrate precisely
  \emph{how} the witness makes the program ``go wrong''.
  %
  We have implemented our system for \ocaml and demonstrate its efficacy
  on a large set of real student programs.
\end{abstract}


% 11 pages total (not including bib)
% ----------------------------------
%  1p : intro
%  2p : overview
%  2p : type-carrying semantics
%  1p : evaluation: quickchecking type errors
%  2p : interactive semantics
%  1p : evaluation: nanomaly
%  1p : related work
%  1p : conclusion

% \section{Introduction}          % 1 page
\section{Introduction}
\label{sec:introduction}

% \paragraph{Problem}
Type errors are a common stumbling block for students trying to learn
typed functional languages like \ocaml and \haskell.
%
The compiler imposes a strict discipline that prevents novices from
running their programs until the type-checker has been appeased.
%
In effect this prioritizes the \emph{static} semantics of the language
over the \emph{dynamic} semantics.
%
This is fine for expert users who are already familiar with the language
and type system, but seems backwards for novices, who are typically
introduced to the dynamic semantics first.

Consider the following ill-typed @factorial@ function, where we return
@true@ in the base case instead of @1@.
%
\begin{code}
  let rec factorial n =
    if n <= 0 then
      true
    else
      n * factorial (n-1)
\end{code}
%
\ocaml's response is to tell us that on line 5, the recursive call to
@factorial@ returns a @bool@, but we needed an @int@ for the
multiplication.
%
\begin{verbatim}
File "fac.ml", line 5, characters 8-23:
Error: This expression has type bool
       but an expression was expected
       of type int
\end{verbatim}
%
\ocaml is correct, but we may ask \emph{why} @factorial@ returns a
@bool@ instead of an @int@.

Contrast this \emph{static} type error to the \emph{dynamic} type error
that arises in the equivalent \ruby function, when we call
@factorial(1)@.
%
\begin{verbatim}
fac.rb:5:in `*': true can't be coerced
                 into Fixnum (TypeError)
  from fac.rb:5:in `factorial'
  from fac.rb:10:in `<main>'
\end{verbatim}
%
The same basic information is there, @*@ was given a @bool@ (@true@)
when it expected an @int@ (@FixNum@), but since the error occurred at
runtime, \ruby is able to provide a stack trace that provides important
contextual information about the error.
%
More importantly, being able to \emph{run} the bad program means we can
also test it on other inputs, \eg @0@, or we can load it into a debugger
and step through the execution to see concretely how the error arises.

To make matters worse, it is well known that type inference algorithms
based on unification have a penchant for \emph{reporting} errors far
from their source.
%
This further increases the novice's confusion and can actively mislead
them to focus their investigation on an irrelevant piece of code.
%
Much work has been done on pinpointing the \emph{source} of a type
error~\cite{lerner_searching_2007,chen_counter-factual_2014,zhang_toward_2014,pavlinovic_finding_2014},
but an accurate source location still does not explain \emph{why} the
program is wrong.

% \paragraph{Contributions}
In this paper we propose exposing \emph{static} type errors for the
\emph{runtime} crashes that they are.
%
Our first contribution is an algorithm for searching for
\emph{witnesses} to type errors, \ie input vectors that cause a program
to crash with a type error.
%
This is not as trivial as it sounds.
%
While one can indeed execute an \ocaml program without type-checking it
-- the dynamic semantics do not depend on the static semantics -- what
types of values should one pass to the program?
%
We do not want to generate \emph{spurious} witnesses, inputs that the
type-checker would have excluded, so we introduce a semantics for
executing \ocaml programs with ``holes'', values whose type we do not
yet know.
%
% A hole can be passed around through the program until it is demanded by
% a primitive operation, \eg addition, function application, branching,
% etc., at which point we know from the context what type it must have,
% and can instantiate it with a concrete value.
%
Our semantics \emph{conservatively} instantiates holes with concrete
values, if we can find a witness to a type error then there is no valid
typing for the program.

Given a witness to a type error, the prospective \ocaml student may
still be at a loss.
%
The standard \ocaml interpreter and debugging infrastructure expect
well-typed programs, so they cannot be used to investigate \emph{how}
the witness causes the program to crash.
%
Thus, our second contribution is an interactive visualization of the
execution of \ocaml programs (well-typed or not).
%
We present the execution as a \emph{reduction graph}, in which the nodes
are expressions and the edges indicate either a single-step reduction or
a sub-term relation.
%
This graph is easy to generate during the execution and admits many of
the standard debugging steps, \eg ``step forward'', ``step into'', and
``step over'', as well as backwards motion and jumping to the next (or
previous) function call.
%
The visualization initially shows the source term (\ie the witness) and
the stuck term (or the final value), and allows the novice to
iteratively expand the computation to show intermediate steps by
selecting an expression and choosing an strategy for picking the next
node to insert into the graph.
%
This process of iterative expansion of the trace ensures that the
context of intermediate steps is not lost; whereas traditional debuggers
force one to move between statements, in our visualization you never
lose sight of where you came from.

\paragraph{Outline}
We begin the rest of the paper with an overview of our search algorithm
and trace visualization (\S~\ref{sec:overview}).
%
% Then, we formalize our algorithm and prove it sound
% (\S~\ref{sec:searching-witness}), and evaluate it on a large set of real
% student programs (\S~\ref{sec:eval}).
% %
% Next, we extend the search algorithm to collect the reduction graph and
% formalize a set of traversals (\S~\ref{sec:explaining}), and measure the
% complexity of these traces on our benchmarks
% (\S~\ref{sec:evaluation-nanomaly}).
Then, we formalize our algorithm and prove it sound
(\S~\ref{sec:searching-witness}).
%
Next, we extend the algorithm to collect a reduction graph and formalize
a set of traversals over the graph (\S~\ref{sec:explaining}).
%
Finally, we evaluate our search algorithm and trace generation on a
large set of real student programs (\S~\ref{sec:evaluation}).
%
% Finally, we present the results of a preliminary user study on the full
% system (\S~\ref{sec:user-study}).

%%% Local Variables:
%%% mode: latex
%%% TeX-master: "main"
%%% End:


% \section{Overview}              % 2 pages?
\section{Overview}
\label{sec:overview}

We start with an overview of our approach to
explaining (static) type errors using \emph{witnesses}
that (dynamically) show how the program goes wrong.
%
We illustrate why generating suitable inputs
to functions is tricky in the absence of type
information.
%
Then we describe our solution to the problem
and highlight the similarity to static type
inference,
%
Finally, we demonstrate our visualization of
the synthesized witnesses.

\subsection{Generating Witnesses}
\label{sec:generating-witnesses}
Our goal is to find concrete values
that demonstrate how a program ``goes wrong''.

\paragraph{Problem: Which inputs are bad?}
%
One approach is to randomly generate input values and
use them to execute the program until we find one that
causes the program to go wrong.
%
Unfortunately, this approach quickly runs aground.
Recall the erroneous @fac@ function from Figure~\ref{fig:factorial}.
%~\S~\ref{sec:introduction}:
%
% \begin{code}
  % let rec fac n =
    % if n <= 0 then
      % true
    % else
      % n * fac (n-1)
% \end{code}
% \ES{having two copies of \texttt{fac} seems silly, but the back-reference across a page boundary is no good either...}
%
What \emph{types} of inputs should we test @fac@ with?
%
Values of type @int@ are fair game, but values of type, say,
@string@ or @int list@ will cause the program to go wrong
in an \emph{irrelevant} manner.
%
Concretely, we want to avoid testing @fac@ with any type other
than @int@ because any other type would cause @fac@ to get stuck
immediately in the @n <= 0@ test.

\paragraph{Solution: Don't generate inputs until forced.}
Our solution is to avoid generating a concrete value for the input at
all, until we can be sure of its type.
%
The intuition is that we want to be as lenient as possible in our tests,
so we make no assumptions about types until it becomes clear from the
context what type an input must have.
%
This is actually quite similar in spirit to type inference.

To defer input generation, we borrow the notion of a ``hole'' from
SmallCheck~\cite{Runciman2008-ka}.
%
A hole --- written \vhole{\thole} --- is a \emph{placeholder} for a
value \ehole of some unknown type \thole.
%
We leave all inputs as uninstantiated holes until they are demanded by
the program, \eg due to a primitive operation like the @<=@ test.

\paragraph{Narrowing Input Types}
Primitive operations, data construction, and case-analysis \emph{narrow}
the types of values.
%
For concrete values this amounts to a runtime type check, we ensure that
the value has a type compatible with the expected type.
%
For holes, this means we now know the type it should
have (or in the case of compound data we know \emph{more} about the
type) so we can instantiate the hole with a value.
%
The value may itself contain more holes, corresponding to components
whose type we still do not know.
%
Consider the @fst@ function:
%
\begin{code}
  let fst p = match p with
    (a, b) -> a
\end{code}
%
The case analysis tells us that @p@ must be a pair, but it says
\emph{nothing} about the contents of the pair.
%
Thus, upon reaching the case-analysis we would generate a pair
containing fresh holes for the @fst@ and @snd@ component.
%
Notice the similarity between instantiation of type variables and
instantiation of holes.
%
We can compute an approximate type for @fst@ by approximating the types
of the (instantiated) input and output, which would give us:
%
\begin{mcode}
  fst : ($\thole_1$ * $\thole_2$) -> $\thole_1$
\end{mcode}
%
We call this type approximate because we only see a single path through
the program, and thus will miss narrowing points that only occur in
other paths.

Returning to @fac@, given a hole as input we will narrow the hole
to an @int@ upon reaching the @<=@ test.
%
At this point we choose a
random @int@\footnote{With standard heuristics~\cite{Claessen2000-lj} to favor small values.}
for the instantiation and
concrete execution takes over entirely, leading us to the expected crash
in the multiplication.

\paragraph{Witness Generality}
We show in \S~\ref{sec:soundness} that our lazy instantiation of holes
produces \emph{general witnesses}.
%
That is, we show that if ``executing''
a function with a hole as input causes the
function to ``go wrong'', then there is
\emph{no possible} type for the function.
%
In other words, for \emph{any} types you might
assign to the function's inputs, there exist values
that will cause the function to go wrong.

\paragraph{Problem: How many inputs does a function take?}
%
There is another wrinkle, though; how did we know
that @fac@ takes a single argument instead of two
(or none)?
%
It is clear, syntactically, that @fac@ takes \emph{at least} one
argument, but in a higher-order language with currying, syntax can be
deceiving.
%
Consider the following definition:
%
\begin{code}
  let incAllByOne = List.map (+ 1)
\end{code}
%
Is @incAllByOne@ a function?
%
If so, how many arguments does it take?
%
The \ocaml\ compiler deduces that @incAllByOne@ takes a single argument
because the \emph{type} of \hbox{@List.map@} says it takes two arguments, and it is
partially applied to @(+ 1)@.
%
As we are dealing with ill-typed programs we do not have the luxury of
typing information.

\paragraph{Solution: Search for saturated application.}
We solve this problem by deducing the number of arguments
via an iterative process. We add arguments one-by-one
until we reach a \emph{saturated} application, \ie\
until evaluating the application returns a value
other than a lambda.

\subsection{Visualizing Witnesses}
\label{sec:visual-witness}
We have described how to reliably find witnesses to type errors in \ocaml,
but this does not fully address our original goal --- to \emph{explain}
the errors.
%
Having identified an input vector that triggers a crash, a common next
step is to step through the program with a \emph{debugger} to observe
how the program evolves.
%
The existing debuggers and interpreters for \ocaml\ assume a type-correct
program, so unfortunately we cannot use them off-the-shelf.
%
Instead we extend our search for witnesses to produce an execution
trace.

\paragraph{Reduction Graph}
Our trace takes the form of a reduction graph, which records small-step
reductions in the context in which they occur.
%
% These graphs have two types of edges:
% %
% (1) ``steps-to'' edges that capture the small-step transition between
% two terms, and
% %
% (2) ``sub-term'' edges that capture the containment relation between two
% terms.
%
For example, evaluating the expression @1+2+3@ would produce the
graph in Figure~\ref{fig:simple-reduction-hi}.
%
\begin{figure}[t]
  \centering
  \includegraphics[height=2in]{simple.png}
  \caption{The reduction graph for \texttt{1+2+3}. The two edges
    produced by the transition from \texttt{1+2+3} to \hbox{\texttt{3+3}}
    are highlighted.}
\label{fig:simple-reduction-hi}
\end{figure}
%
Notice that when we transition from @1+2+3@ to @3+3@ we collect
both that edge \emph{and} an edge from the sub-term @1+2@ to @3@.
%
These additional edges allow us to implement two common debugging
operations \emph{post-hoc}: ``step into'' to zoom in on a specific
function call, and ``step over'' to skip over an uninteresting
sub-computation.

\paragraph{Interacting with the graph}
The reduction graph is useful for formulating and executing traversals,
but displaying it all at once would quickly become overwhelming.
%
Our interaction begins by displaying a big-step reduction, \ie the
witness followed by the stuck term.
%
The user can then progressively fill in the hidden steps of the
computation by selecting a visible term and choosing one of the
applicable traversal strategies --- described in
\S~\ref{sec:interactive} --- to insert another term into the
visualization.

\paragraph{Jump-compressed Witnesses}
It is rare for the initial state of the visualization to be
informative enough to diagnose the error.
%
Rather than abandon the user, we provide a short-cut to expand the witness
to a \emph{jump-compressed} trace, which contains every function call
and return step.
%
The jump-compressed trace abstracts the computation as a sequence of
call-response pairs, providing a high-level overview of steps taken
to reach the crash, and a high level of compression compared to the
full trace.
%
For example, the jump-compressed trace in Figure~\ref{fig:factorial}
contains 4 nodes compared to the 19 in the fully expanded trace.
%
Our benchmark suite of student programs shows that jump-compression is
practical, with an average jump-compressed trace size of 7 nodes and a
median of 5.

% A sample interaction with the trace of @fac 1@ can be seen in
% Figure~\ref{fig:nanomaly-factorial}.
% %
% % \begin{figure*}[t]
% % \centering
% % \includegraphics[width=0.8\linewidth]{fac-steps.png}
% % \caption{A sequence of interactions with the trace of
% %   \texttt{fac 1}. The stuck term is red, in each node the redex is
% %   highlighted. Thick arrows denote a multi-step transition, thin arrows
% %   denote a single-step transition. We start in step 1. In step 2 we jump
% %   forward from the witness to the next function call. In step 3 we step
% %   into the recursive \texttt{fac 0} call, which spawns a new ``thread''
% %   of execution. In step 4 we take a single step forward from
% %   \texttt{fac 0} (hiding the context for space).}
% % \label{fig:nanomaly-factorial}
% % \end{figure*}
% %
% The initial state of the visualization tells us that after some number
% of steps -- the thick arrow denotes a multi-step transition -- we try to
% multiply @1@ by @true@.

% Upon seeing the stuck term, we might wonder where @true@ came from.
% %
% To investigate we select the stuck term and click the ``jump backward''
% button to search backwards from the stuck term for the most recent
% function call, which brings us to @1 * fac 0@. Notice at this point that
% @fac 0@ is highlighted while @1 *@ is grayed out. This tells us that
% @fac 0@ is the redex in this term.

% @fac 0@ seems like the right thing to do so we choose to ``step into''
% it, which inserts a new multi-step transition from @fac 0@ to @true@.
% %
% Finally, we take a ``step forward'' from \hbox{@fac 0@,} bringing us to @fac@'s
% body. If we mouse over the body term we will see a popup with the
% environment at this point, notably telling us that @n = 0@. At this point
% it is clear that @fac@ handles the @n <= 0@ case incorrectly and should
% instead return an @int@.

% Upon seeing the stuck term, one might wonder where the @function@
% came from.
% %
% To investigate we select the stuck term and click the ``jump backward''
% button to search backwards from the stuck term for the most recent
% function call, which brings us to @listReverse [] = w@, in the same
% context as before.
% %
% Uncontent with the explanation so far, we ``step forward'' twice from
% the @listReverse []@ term, bringing us to @helper [] = w@.
% %
% At this point it is clear that the @helper@ function is not defined
% correctly, we have supplied it with the single argument we expected and
% yet it still returned a @function@.

% The problem is that the @function@ keyword in \ocaml defines an
% anonymous function that takes a single argument and immediately does a
% case-analysis without giving the argument a name.
% %
% The solution is to replace @function@ with an explicit @match xs with@
% -- naming the value we wish to case-analyse.
% %
% After applying our fix, \nanomaly -- and more importantly \ocaml --
% decide that @listReverse@ is safe to run.
% \ES{these last few paragraphs probably belong in the overview}
%





%%% Local Variables:
%%% mode: latex
%%% TeX-master: "main"
%%% End:


% \section{Type-carrying Semantics} % 2 pages
\section{Searching for Type-Error Witnesses}
\label{sec:searching-witness}
% \begin{itemize}
% \item how do we run ill-typed programs?
% \item for a lang like ocaml, dynamic semantics are independent of static
%   semantics, just lambda calculus. so no problem to run ill-typed
%   program
% \item but what about functions? what type of arguments should we pass? consider
%
% \begin{lstlisting}
% let f x =
%   let y = 1 + x in
%     1. +. y
% \end{lstlisting}
%
% does \texttt{f} take an int, float, string? int and float are both
% somewhat plausible, but string or anything else is ``clearly'' bogus. so
% we cannot provide \emph{completely arbitrary} inputs to
% \texttt{f}. Instead, we call \texttt{f} with a \emph{hole}, written
% \ehole{}, which is a placeholder for a value whose type we have not
% yet determined. As we execute the program, we instantiate holes with
% concrete values as demanded by the primitive operations in the
% program. For example, the hole we pass to f will be instantiated to an
% int when we reach the \lstinline{1 + x} term. Thus, y will be an int as
% well, and the program will get stuck at \lstinline{1. +. y}. \ES{this
%   reads more like overview text..}
%
% % \item values are tagged with their types, just like ``untyped'' langs
% % \item special ``hole'' value whose type is not yet known, used for function args
% % \item on-the-fly unification to determine ``correct'' type for holes
% \end{itemize}
%
Next, we formalize our search for witnesses to type-errors.
%
We present the syntax and operational semantics of \lang -- a simple
lambda calculus with integers and booleans, extended with our notion of
holes -- as well as our search algorithm.
%
We prove that our system \emph{soundly} finds witnesses, \ie if we find
a witness then there is no possible typing for the input program.
%
\subsection{Syntax}
\label{sec:syntax}
\begin{figure}
% \hrule width 0.48\textwidth \vspace{0.05in}
$$
\begin{array}{rrcl}
\emphbf{Expressions} \quad
  & e & ::=    & v \spmid x \spmid \eapp{e}{e} \spmid \eplus{e}{e}\\
  &   & \spmid & \eif{e}{e}{e} \\
  &   & \spmid & \elet{x}{e}{e} \\
  &   & \spmid & \edcapp{c}{e} \spmid \ematch{e}{\ealt{\edcapp{c}{x}}{e}} \\[0.05in]

\emphbf{Values} \quad
  & v & ::= &  n \spmid b \spmid \efun{x}{e} \spmid \ehole{n} \spmid \stuck \\[0.05in]

\emphbf{Integers} \quad
  & n & ::= &  0,1,-1,\ldots \\[0.05in]

\emphbf{Booleans} \quad
  & b & ::= &  \etrue \spmid \efalse \\[0.05in]

\emphbf{Types} \quad
  & t & ::= & \tbool \spmid \tint \spmid \tfun{t}{t} \spmid \thole{n} \\[0.05in]

\emphbf{Substitutions} \quad
  & \vsu & ::= & \emptysu \spmid \mksub{n}{v}; \vsu \\[0.1in]
\end{array}
$$
% \hrule width 0.48\textwidth

$$
\begin{array}{rrcl}
\emphbf{Contexts} \quad
  & C
  & ::=
  &   	 \bullet
  \spmid \eapp{C}{e}
  \spmid \eapp{v}{C} \\
  & & \spmid & \eplus{C}{e} \spmid \eplus{v}{C} \\
  & & \spmid & \eif{C}{e}{e} \\
  & & \spmid & \elet{x}{C}{e}
  \\[0.05in]
\end{array}
$$

% \judgementHead{Reduction}{\eval{e}{e}}

% $$
% \begin{array}{rcl}
% \eval{C[e]&}{&C[e']} \qquad \text{if}\ \eval{e}{e'} \\
% 	\eval{\eapp{c}{v}&}{& \ceval{c}{v}}\\
% \eval{\eapp{(\efun{x}{\tau_x}{e})}{e_x}&}{&e\sub{x}{e_x}}\\
% 	\eval{\elet{x}{e_x}{e}&}{&e\sub{x}{e_x}} \\
% 	\eval{\ecase{D_j\ \overline{e}}{D_i}{\overline{y_i}}{e_i}{x}&}
% 	{&e_j\sub{x}{D_j\ \overline{e}}\sub{\overline{y_j}}{\overline{e}}} \\
% \end{array}
% $$

\caption{Syntax of \lang}
\label{fig:syntax}
\end{figure}

%
Figure~\ref{fig:syntax} describes the syntax of \lang, a simple lambda
calculus with numbers and booleans.
%
As we are specifically interested in
programs that \emph{do} go wrong, we include an explicit \stuck state in
our syntax.
%
\paragraph{Holes}
\label{sec:holes}
The main novelty in our system is the notion of a ``hole'', written
\ehole{}, which represents an uninstantiated value.
%
Importantly, we do not even know what type the value should have.
%
Holes may also appear in types, where they may be thought of as type
variables that we will not generalize over.
%
\subsection{Semantics}
\label{sec:semantics}
\begin{figure}
\judgementHead{Evaluation}{\step{e}{\su}{e}{\su}}
$$
\inference[\recontext]
  {\step{e}{\su}{e_1}{\su_1}}
  {\step{C[e]}{\su}{C[e_1]}{\su_1}}
$$

$$
\inference[\restuck]
  {}
  {\step{C[\stuck]}{\su}{\stuck}{\su}}
$$

$$
\inference[\replusgood]
  {\pair{n_1}{\su_2} = \force{v_1}{\tnat} \\
   \pair{n_2}{\su_3} = \force{v_2}{\tnat} \\ 
   n = \eplus{n_1}{n_2}}
  {\step{\eplus{v_1}{v_2}}{\su_1}{n}{\su_1\su_2\su_3}}
$$

$$
\inference[\replusbadone]
  {\pair{\stuck}{\su_2} = \force{v_1}{\tnat}}
  {\step{\eplus{v_1}{v_2}}{\su_1}{\stuck}{\su_1\su_2}}
$$

$$
\inference[\replusbadtwo]
  {\pair{\stuck}{\su_2} = \force{v_2}{\tnat}}
  {\step{\eplus{v_1}{v_2}}{\su_1}{\stuck}{\su_1\su_2}}
$$

$$
\inference[\reifgoodone]
  {\pair{\etrue}{\su_2} = \force{v}{\tbool}}
  {\step{\eif{v}{e_1}{e_2}}{\su_1}{e_1}{\su_1\su_2}}
$$

$$
\inference[\reifgoodtwo]
  {\pair{\efalse}{\su_2} = \force{v}{\tbool}}
  {\step{\eif{v}{e_1}{e_2}}{\su_1}{e_2}{\su_1\su_2}}
$$

$$
\inference[\reifbad]
  {\pair{\stuck}{\su_2} = \force{v}{\tbool}}
  {\step{\eif{v}{e_1}{e_2}}{\su_1}{\stuck}{\su_1\su_2}}
$$

$$
\inference[\reappgood]
  {\pair{\efun{x}{e}}{\su_2} = \force{v_1}{\tfun{\thole{}}{\thole{}}}}
  {\step{\eapp{v_1}{v_2}}{\su_1}{e\sub{x}{v_2}}{\su_1\su_2}}
$$

$$
\inference[\reappbad]
  {\pair{\stuck}{\su_2} = \force{v_1}{\tfun{\thole{}}{\thole{}}}}
  {\step{\eapp{v_1}{v_2}}{\su_1}{\stuck}{\su_1\su_2}}
$$

$$
\inference[\relet]
  {}
  {\step{\elet{x}{v}{e}}{\su}{e\sub{x}{v}}{\su}}
$$
\\ [0.1in]

\relDescription{\forcesym and \gensym}

$$
\begin{array}{lcl}
\force{\ehole{i}}{t} & \defeq & \elet{v}{\gen{t}}{\pair{v}{\ehole{i} \mapsto v}} \\
\force{v}{\ehole{}}  & \defeq & \pair{v}{\emptysu} \\
\force{n}{\tnat}    & \defeq & \pair{n}{\emptysu} \\
\force{v}{\tnat}    & \defeq & \pair{\stuck}{\emptysu} \\
\force{b}{\tbool}   & \defeq & \pair{b}{\emptysu} \\
\force{v}{\tbool}   & \defeq & \pair{\stuck}{\emptysu} \\
\force{\efun{x}{e}}{\tfun{\thole{}}{\thole{}}} & \defeq & \pair{\efun{x}{e}}{\emptysu} \\
\force{v}{\tfun{\thole{}}{\thole{}}} & \defeq & \pair{\stuck}{\emptysu} \\
\end{array}
$$
$$
\begin{array}{lcll}
\gen{\tnat}   & \defeq & n & \\
\gen{\tbool}  & \defeq & b & \\
\gen{\tfun{t_1}{t_2}} & \defeq & \efun{x}{\ehole{i}}, & \quad \text{$i$ is fresh} \\
\gen{\thole{}} & \defeq & \ehole{i}, & \quad \text{$i$ is fresh} \\
\end{array}
$$


\caption{Evaluation relation}
\label{fig:evaluation}
\end{figure}

%
Figure~\ref{fig:operational} describes the small-step contextual
reduction semantics for \lang.
%
We write \stepi{j}{e}{\su}{e'}{\su'} if there exist $e_1,\ldots,e_j$ such that
$e$ is $e_1$, $e'$ is $e_j$ and $\forall i,j, 1 \leq i < j$, we have
\step{e_i}{\su_i}{e_{i+1}}{\su_{i+1}}.
%
We write \steps{e}{\su}{e'}{\su'} if there exists some (finite) $j$ such that
$\stepi{j}{e}{\su}{e'}{\su'}$.
%
The evaluation relation is parameterized by a pair of \forcesym and
\gensym functions that handle runtime type-checking and instantiation of
holes respectively.
%
The relation must also maintain a substitution \su, mapping holes to
concrete values, so that we do not instantiate the same hole with
different values in different contexts, and so that we can report a
concrete witness to any discovered type errors.

Note that each primitive reduction step -- addition, if-elimination, and
function application -- uses \forcesym to ensure that values have the
appropriate type (and that holes are instantiated) before continuing the
computation.
%
Additionally, beta-reduction \emph{does not} type-check its argument, it
only ensures that the value being applied is a function.
%
\begin{thm}
\label{thm:all-reduce}
  Every closed expression $e$ reduces to a value $v$ (which may be \stuck).
  \ES{do we really need to state this, or is it obvious?}
\end{thm}
% \begin{proof}%[Proof of \autoref{thm:all-reduce}]
%   Simple induction on the evaluation relation.
% \end{proof}
%
\subsection{Soundness}
\label{sec:soundness}
We now show that our evaluation relation conservatively instantiates
holes, \ie if, given a function $f$, we report that
$\steps{\eapp{f}{\ehole{}}}{\emptysu}{\stuck}{\su}$,
then for any type $\tfun{s}{t}$ you assign to $f$, there exists an input \hastype{v}{s} such that
$\steps{\eapp{f}{v}}{\emptysu}{\stuck}{\su}$.

We can think of the evaluation of \eapp{f}{\ehole{}} as computing a
\emph{partial type} -- a type that may contain holes -- for $f$.
%
We can extract this type from the result of evaluation as follows.
%
\begin{defn}
\label{defn:partial-type}
  If \stepi{i}{\eapp{f}{\ehole{}}}{\emptysu}{e}{\su}, then the $i$th
  \emph{partial type} of f, written \ptype{i}{f},
  is \tfun{\typeof{\subst{\su}{\ehole{}}}}{\typeof{e}}.

  We will omit the subscript when we wish to refer to the final partial
  type.
\end{defn}
%
The \typeof{} function is an approximation of the type of an expression.
\begin{defn}
\label{def:typeof}
  \[
  \begin{array}{lcll}
    \typeof{n}   & \defeq & \tint & \\
    \typeof{b}   & \defeq & \tbool & \\
    \typeof{\efun{x}{e}} & \defeq & \tfun{\thole{i}}{\typeof{e}}, & \quad \text{$i$ is fresh} \\
    \typeof{e} & \defeq & \thole{i}, & \quad \text{$i$ is fresh} \\
  \end{array}
  \]
\end{defn}
%
We also define a compatibility relation between types.
%
\begin{defn}
\label{defn:type-compat}
  A type $s$ is \emph{compatible} with a type $t$, written \tcompat{s}{t}, if
  $\exists \su.\ t = \subst{\su}{s} \lor s = \subst{\su}{t}$.
  \ES{we're abusing the \su notation here..}
\end{defn}
%
Given these two definitions, we show that each evaluation step
refines the partial type of $f$.
%
\begin{lem}
\label{lem:refine-partial}
  For all $k$, \tcompat{\ptype{k}{f}}{\ptype{k+1}{f}}.
\end{lem}
\begin{proof}
  By case analysis on the evaluation rules.
  %
  Note that all rules preserve partial types with the exception of when
  \forcesym is called on a hole, which case we instantiate the hole with
  a concrete value.
  %
  But \hastype{\ehole{}}{\thole{}}, which is compatible with any type.
\end{proof}
%
Furthermore, only a call to \forcesym can change the partial type of $f$.
%
\begin{lem}
\label{lem:force-inst}
  For all $k$, if $\ptype{k}{f} \neq \ptype{k+1}{f}$, then \forcesym must
  have been called at step $k+1$.
\end{lem}
\begin{proof}
  By case analysis on the evaluation rules.
  %
  If $\ptype{k}{f} \neq \ptype{k+1}{f}$ then one of the holes in $f$'s
  argument must have been instantiated with a concrete value at step
  $k+1$.
  %
  An examination of the rules shows that only place this happens is
  in the first case of \forcesym.
\end{proof}
%
Finally, we show that any value that is not compatible with the $k$th
partial type of $f$ will cause $f$ to get stuck in \emph{at most} $k$
steps.
%
\begin{lem}
\label{lem:k-stuck}
  For any \hastype{v}{t} that is not compatible with
  the input of \ptype{k}{f}, \stepi{k}{\eapp{f}{v}}{\emptysu}{\stuck}{\su}.
\end{lem}
\begin{proof}
  By induction on $k$.

  Let $\tfun{s_k}{\_} = \ptype{k}{f}$. Suppose \hastype{v}{\tincompat{t}{s_{k+1}}}, we
  will show that \stepi{k+1}{\eapp{f}{v}}{\emptysu}{\stuck}{\su}.
  \begin{description}
  \item[Case \tincompat{t}{s_k}:]
    The inductive hypothesis applies.
  \item[Case \tcompat{t}{s_k} but \tincompat{t}{s_{k+1}}:]
    By Lemma~\ref{lem:force-inst} we must have called \forcesym at step
    $k+1$.
    %
    A case analysis of the applicable rules shows that \forcesym cannot
    have succeeded.
  \end{description}
\end{proof}
%
Now we can prove our soundness theorem.
%
\begin{thm}
\label{thm:soundness}
  For any function $f$, if \steps{\eapp{f}{\ehole{}}}{\emptysu}{\stuck}{\su},
  then $\forall t. \exists \hastype{v}{t}. \steps{\eapp{f}{v}}{\emptysu}{\stuck}{\su}$.
\end{thm}
\begin{proof}
  Let $\tfun{s}{\_} = \ptype{}{f}$.
  \begin{description}
  \item[Case \tcompat{s}{t}:] Our witness is already valid.
  \item[Case \tincompat{s}{t}:] We can apply Lemma~\ref{lem:k-stuck} and
    use \emph{any} \hastype{v}{t} as a witness for $f$.
  \end{description}
\end{proof}

\subsection{Search Algorithm}
\label{sec:search-algorithm}
%
We have shown how to find a witness for a function of a single argument,
but in a language with higher-order functions and currying it may not be
clear \emph{syntactically} how many arguments a function takes.
%
Thus, we wrap our operational semantics for \lang in a search loop that
supplies an increasing number of arguments until the function returns a
value that is not a lambda.

The top-level search loop takes as input an open program -- a sequence
of binders -- and searches for an expression that closes the input
program and gets stuck. Concretely, given an input
%
\begin{code}
  let f1 = e1 in
  let f2 = e2 in
  ...
  let fn = en in
\end{code}
%
we will search for an expression of the form:
%
\begin{code}
  fn v1 ... vn
\end{code}
%
Figure~\ref{fig:expression-api} describes a small API for manipulating
and evaluating \lang expressions, which we will use to define our search
algorithm.
%
The bulk of the search is performed by @eval@, which
implements our operational semantics from \S\ref{sec:semantics}.
%
The operational semantics is non-deterministic due to \gensym,
thus @eval@ returns a list of possible results.
%
\begin{figure}[t]
  \centering
  \begin{mcode}
  -- transitive small-step reduction,
  -- returning a list of results
  eval :: ($e$, $\su$) -> [($v$, $\su$)]

  -- manipulating expressions
  subst   :: $\su$ -> $v$ -> $v$
  mkApps  :: $e$ -> [$e$] -> $e$
  mkLets  :: [($x$, $e$)] -> $e$ -> $e$
  isStuck :: $v$ -> Bool
  \end{mcode}
  \caption{Expression API}
  \label{fig:expression-api}
\end{figure}
%
We also define a few helper functions for manipulating expressions:
\begin{itemize}
\item @subst@ applies a substitution of holes to a value,
\item @mkApps@ creates a nested sequence of applications in the usual
  left-associative style,
\item @mkLets@ takes a list of binders and a body expression, and
  creates a sequence of nested let-binders, and
\item @isStuck@ tests whether a value is the \stuck term.
\end{itemize}
%
\begin{figure*}[t]
  \centering
  \begin{mcode}
  check :: [($x$, $e$)] -> Result
  check bnds =
    -- (2) search for a witness
    case find (isStuck . fst) results of
      Nothing      -> Safe
      Just (_, su) -> Unsafe (mkApps f (subst su args))
    where
      (args, results) = loop []
      f               = snd (last bnds)
      build args      = mkLets bnds (mkApps f args)

      -- (1) find the correct number of arguments
      loop :: [$v$] -> ([$v$], [($v$, $\su$)])
      loop args = case eval (build args, []) of
        ($\efun{x}{e}$, _) : _ -> loop (args `snoc` $\ehole{}$)
        results      -> (args, results)
  \end{mcode}
  \caption{Implementing our counter-example search in terms of the
    operational semantics.
    \ES{1 and 2 are backwards, ugh...}
    \ES{should address case where output types of successive runs dont match}
  }
  \label{fig:search-algo}
\end{figure*}
%
Figure~\ref{fig:search-algo} summarizes the overall implementation of
our search for witnesses, which takes as input a sequences of binders
and returns either @Safe@ if no witness could be found, or |Unsafe $e$|
where $e$ is a term that causes the input program to get stuck.
%
The search is split into two phases: (1) we supply an increasing number
of arguments until we find a saturated application, and (2) we search
through the list of results from the non-deterministic evaluator for a
witness.


% Algorithm:
%   Input: ML Program (let f1 = e1 in let f2 = e2 in ... e)
%   Output: 'safe' or 'fn v1 .. vn' where 'fn v1 .. vn' gets stuck


% \section{Evalution: Recasting Type Errors as Runtime Errors} % 1 page
% \section{Evaluation: Recasting Type Errors as Runtime Errors}
\label{sec:eval-witness}
%
The immediate question is ``What fraction of type errors admit
witnesses?''
%
To answer this question we implemented a prototype of our search
procedure for the pure subset of \ocaml, \ie \lang extended with
algebraic datatypes and records. 
%
In our implementation we instantiated \gensym with a simple random
generation of values, which we will show is more than sufficient for the
majority of type errors.
%
We evaluated our implementation of two sets of known-bad programs, \ie
programs that were rejected by the \ocaml compiler because of a type
error.
%
The first dataset comes from the Spring 2014 undergraduate Programming
Languages course at UC San Diego (IRB \#XXXXX). 
%
We recorded each interaction with the \ocaml top-level system over the
course of the first three assignments, from which we extracted XXXX
distinct, ill-typed \ocaml programs.
%
The second dataset -- widely used in the literature -- comes from a
similar course at the University of Washington~\cite{XXXXX}, from which
we extracted XXXX ill-typed programs.

We ran our search algorithm on each program with the entry point set to
the function that \ocaml had identified as containing a type error. 
%
Due to the possibility of non-termination we set a limit on the number
of reductions to perform, increasing in 1,000-step increments from 1,000
steps to 10,000 steps total.
%
We also added a na\"ive check for infinite recursion; at each recursive
function call we check whether the new arguments are identical to the
current arguments.
%
If so, the function cannot possibly terminate and we report an error.
%
While not a \emph{type error}, infinite recursion is still a clear bug
in the program, and thus valuable feedback for the user.

\begin{figure}[t]
  \centering

  \includegraphics[width=\linewidth]{coverage.png}
  \includegraphics[width=\linewidth]{distrib_seminal.png}

% \begin{tikzpicture} 
% \begin{axis}[
%   title=Cumulative Coverage,
%   ybar interval,
%   xticklabel=
% \pgfmathprintnumber\tick--\pgfmathprintnumber\nexttick
% ]
%     \addplot+[hist={bins=3, cumulative}]
%         table[row sep=\\,y index=0] {
%         data\\
%         1\\ 2\\ 1\\ 5\\ 4\\ 10\\
%         7\\ 10\\ 9\\ 8\\ 9\\ 9\\
%     };
% \end{axis}
% \end{tikzpicture}

% \begin{tikzpicture} \begin{axis}[
%   title=Result Distribution,
% %  ybar interval,
% %  symbolic hist/data coords={S,U,B,O,D,T},
%   hist/symbolic coords={S,U,B,O,D,T},
% %  xticklabel={[\tick--\nexttick[}],
% ]
%     \addplot+[hist={bins=3}]
%         table[row sep=\\,y index=0] {
%         data\\
%         S\\ S\\ U\\
%     };
% \end{axis}
% \end{tikzpicture}

  Have a table or graph! \ES{graph of cumulative errors per timeout? \ie
    monotonically increasing, with a cap of 90\% or so}
\caption{Experimental results}
\label{fig:results-witness}
\end{figure}

\paragraph{Results}
\label{sec:results-witness}
The results of our experiments are summarized in
Figure~\ref{fig:results-witness}.
%
In both datasets our tool was able to find a witness for at least XX\%
of the programs.
%
Interestingly, while the vast majority of witnesses corresponded to a
type-error, as expected, X\% triggered an unbound variable error and X\%
triggered an infinite recursion error.
%
XX programs were deemed safe and XX timed out even at 10,000 steps, \ie
we could not provide any useful feedback for XX\% of the total programs.
%
While a more advanced search procedure, \eg dynamic-symbolic execution,
could likely trigger more of the type errors, our experiments show that
type errors are coarse enough (or that novice programs are \emph{simple}
enough) that these techniques are not necessary.



% \begin{itemize}
% \item benchmarks: our data + seminal data
% \item both cases: \textbf{random} search sufficient to trigger runtime crash in 80\% of programs
% \item how many of the ``safe'' programs are actually safe??
% \end{itemize}


% \section{``Interactive'' Semantics \& Trace Exploration} % 2 pages
\section{Explaining Type Errors With Traces}
\label{sec:explaining}

% We have shown how to reliably find witnesses to type errors in \ocaml,
% but this not fully address our original goal -- to \emph{explain} the
% errors.
% %
% Having identified an input vector that triggers a bug, a common next
% step is to step through the program with a \emph{debugger} to observe
% how the program evolves.
% %
% The existing debuggers and interpreters for \ocaml assume a type-correct
% program, so unfortunately we cannot use them off-the-shelf.
% %
% Instead we return to our semantics for \lang and extend the evaluation
% rules to collect a trace that we can present to the user, demonstrating
% precisely \emph{how} their program went wrong.

% The trace takes the form of a \emph{reduction graph}, where the nodes
% are terms and the edges represent either the single-step
% $\hookrightarrow$ or a ``sub-term'' relation. For example, evaluating
% the expression @1 + 2 + 3@ would produce the graph in
% Figure~\ref{fig:simple-reduction}.
% %
% \begin{figure}[t]
%   \centering
%   \includegraphics[width=\linewidth]{simple.png}
% \caption{The reduction graph for \texttt{1 + 2 + 3}.}
% \label{fig:simple-reduction}
% \end{figure}
% %
% We choose this graph representation instead of a simple, linear sequence
% of expressions because it will allow us to express a variety of
% traversals, such as ``step into'' and ``step over'' -- commonly found in
% traditional debuggers.
>>>>>>> 8bffd296d9a45d936befb58888432d3d99153b97

In this section we formalize the collection and traversal of the reduction graph.
%
First, we extend our semantics from \S~\ref{sec:semantics} to collect
the edges in the graph (\S~\ref{sec:inter-semant}).
%
Then, we express a set of common debugging steps as graph traversals
(\S~\ref{sec:traversing-graph}).

\subsection{Tracing Semantics}
\label{sec:inter-semant}
%
\begin{figure*}[t]
\relDescription{Trace Syntax}
$$
\begin{array}{rrcl}
  & \tr & ::= & \bullet \spmid \singlestep{e}{e}; \tr \spmid \subterm{e}{e}; \tr
\end{array}
$$
\\
\relDescription{\subtermssym}
\begin{gather*}
\begin{array}{lcl}
\subtermssym                 & \dcolon & e \to \tr \\
\subterms{\eapp{e_1}{e_2}}   & \defeq & \subterm{\eapp{e_1}{e_2}}{e_1}; \subterm{\eapp{e_1}{e_2}}{e_2} \\
\subterms{\eplus{e_1}{e_2}}   & \defeq & \subterm{\eplus{e_1}{e_2}}{e_1}; \subterm{\eplus{e_1}{e_2}}{e_2} \\
\subterms{\eif{e_1}{e_2}{e_3}}   & \defeq & \subterm{\eif{e_1}{e_2}{e_3}}{e_1}; \\
                                &        & \subterm{\eif{e_1}{e_2}{e_3}}{e_2}; \\
                                &        & \subterm{\eif{e_1}{e_2}{e_3}}{e_3} \\
\subterms{\elet{x}{e_1}{e_2}}   & \defeq & \subterm{\elet{x}{e_1}{e_2}}{e_1}; \\
                                &        & \subterm{\elet{x}{e_1}{e_2}}{e_2} \\
\subterms{\efun{x}{e}}       & \defeq & \subterm{\efun{x}{e}}{e} \\
\subterms{e}                 & \defeq & \bullet
\end{array}
\end{gather*}
\judgementHead{Evaluation}{\stepg{e}{\su}{\tr}{e}{\su}{\tr}}
\begin{gather*}
\inference[\recontext]
  {\stepg{e}{\su}{\tr}{e_1}{\su_1}{\tr_1}}
  {\stepg{C[e]}{\su}{\tr}{C[e_1]}{\su_1}{\singlestep{C[e]}{C[e_1]}; \subterms{C[e_1]}; \tr_1}}
\\ \\
\inference[\reappgood]
  {\pair{\efun{x}{e}}{\su_2} = \force{v_1}{\tfun{\thole{}}{\thole{}}}{\su_1}}
  {\stepg{\eapp{v_1}{v_2}}{\su_1}{\tr}
         {e\sub{x}{v_2}}{\su_1;\su_2}{\singlestep{\eapp{v_1}{v_2}}{e\sub{x}{v_2}}; \subterms{e\sub{x}{v_2}}; \tr}}
\\ \\
\inference[\reappbad]
  {\pair{\stuck}{\su_2} = \force{v_1}{\tfun{\thole{}}{\thole{}}}{\su_1}}
  {\stepg{\eapp{v_1}{v_2}}{\su_1}{\tr}{\stuck}{\su_1;\su_2}{\singlestep{\eapp{v_1}{v_2}}{\stuck}; \tr}}
\end{gather*}
\caption{A selection of the operational semantics from
  Figure~\ref{fig:operational}, extended to collect a full reduction
  graph.}
\label{fig:interactive}
\end{figure*}
%
The changes to the operational semantics of \S~\ref{sec:semantics} are
mechanical, so we will not reproduce them in full; instead we describe
the procedure for extending a transition rule and provide a selection of
examples in Figure~\ref{fig:interactive}.

First, we extend the transition relation to collect a set of edges from
which we will construct the graph.
%
An edge between two expressions is either a ``steps-to'' edge -- written
\singlestep{e_1}{e_2} -- indicating that $e_1$ transitions to $e_2$ in a
single step, or a ``sub-term'' edge -- written \subterm{e_1}{e_2} --
indicating that $e_1$ contains $e_2$ as a sub-expression.
%
Collecting the steps-to edges is a simple matter of recording the
consequent of each original rule in the trace; each original judgment
\step{e_1}{\su_1}{e_2}{\su_2} becomes
\stepg{e_1}{\su_1}{\tr_1}{e_2}{\su_2}{\singlestep{e_1}{e_2}; \tr_1}.
%
The sub-term edges can be delegated to a new \subtermssym helper
function, which adds edges from an expression to each of its
\emph{immediate} sub-expressions.
%
We collect new sub-term edges after each transition, thus the final
template for the small-step relation is:
\[
\stepg{e_1}{\su_1}{\tr_1}{e_2}{\su_2}{\singlestep{e_1}{e_2}; \subterms{e_2}; \tr_1}
\]

After evaluation the reduction graph can be constructed directly from
the trace $\tr$ as follows:
\[
G(\tr) = \pair{\{e \spmid e \in \tr\}}{\tr}
\]

\subsection{Traversing the Reduction Graph}
\label{sec:traversing-graph}
Given a reduction graph $G$ and an initial path
$p = \singlestep{e_1}{\singlestep{e_2}{e_n}}$, such that
\stepsg{e_1}{\emptysu}{\bullet}{e_n}{\su}{\tr}, we define the following
traversals in Figure~\ref{fig:traversing-graph}:
%
\begin{itemize}
\item \stepforwardsym takes a single step forward.
\item \stepbackwardsym takes a single step backward.
\item \jumpforwardsym takes a ``big'' step forward to the next function call.
\item \jumpbackwardsym takes a ``big'' step backward to the previous function call.
\item \stepintosym steps into a function call in a sub-term, isolating it from the context.
\item \stepoversym steps over a function call in a sub-term.
\end{itemize}
%
The initial path $p$ is required for the backward-steps as a node may
have multiple incoming $\leadsto$ edges, \eg
\singlestep{\eplus{1}{2}}{3} and \singlestep{\eplus{2}{1}}{3}.
%
The sub-term edges $\searrow$ allow us to decompose an expression into a
sub-expression and the surrounding context, thus enabling the \stepintosym
and \stepoversym traversals.
%


\begin{figure*}[t]
\centering
\[
\begin{array}{lcl}
\stepforward{G}{p}{e_i}  &\defeq& \left\{\begin{array}{ll}
    e_j, & \text{where } \singlestep{e_i}{e_j} \in G
                         \end{array}\right\} \\ \\
\stepbackward{G}{p}{e_i}  &\defeq& \left\{\begin{array}{ll}
    e_j, & \text{where } \singlestep{e_j}{e_i} \in G \text{ and } e_j \in p
                         \end{array}\right\} \\ \\
\jumpforward{G}{p}{e_i} &\defeq& \text{let } e_j = \stepforward{G}{p}{e_i} \text{ in }
                         \left\{\begin{array}{ll}
                         e_j, & \text{if } e_j = \eapp{v_1}{v_2} \\
                         \jumpforward{G}{p}{e_{j}}, & \text{otherwise}
                         \end{array}\right\} \\ \\
\jumpbackward{G}{p}{e_i} &\defeq& \text{let } e_j = \stepbackward{G}{p}{e_i} \text{ in }
                         \left\{\begin{array}{ll}
                         e_j, & \text{if } e_j = \eapp{v_1}{v_2} \\
                         \jumpbackward{G}{p}{e_{j}}, & \text{otherwise}
                         \end{array}\right\} \\ \\
\stepinto{G}{p}{e_i} &\defeq& \left\{\begin{array}{ll}
                         e\sub{x}{v_2}, & \text{if } e_i = C[\eapp{v_1}{v_2}] \text{ and } \singlestep{\eapp{v_1}{v_2}}{e\sub{x}{v_2}}
                         \end{array}\right\} \\ \\
\stepover{G}{p}{e_i} &\defeq& \left\{\begin{array}{ll}
                         C[v], & \text{if } e_i = C[\eapp{v_1}{v_2}] \text{ and } \multistep{\eapp{v_1}{v_2}}{v}
                         \end{array}\right\}
\end{array}
\]
\caption{Rules for traversing the reduction graph given a path and
  node. \stepintosym and \stepoversym require a traversal of the
  sub-term edges to decompose $e_i$ into the target expression
  \eapp{v_1}{v_2} and the context $C$.  \ES{these rules are quite ugly and waste space..} }
\label{fig:traversing-graph}
\end{figure*}


% \begin{itemize}
% \item extend operational semantics to collect reduction graph
% \item nodes are terms, edges indicate ``steps-to'' and ``sub-term'' relations
% \item visualize path through reduction graph
% \item expand edges to reveal more fine-grained steps (step/jump forward/backward)
% \item never lose context (unlike traditional debugger)
% \end{itemize}


% \section{Our Implementation: \nanomaly}
\label{sec:impl-nanomaly}
We have implemented our interactive semantics for the pure subset of
\ocaml with a web-based visualization of the generated traces in a tool
called \nanomaly.
%
It takes as input an \ocaml program and a function to check, searches
for a witness as described in \S~\ref{sec:searching-witness}, and
produces as output an interactive trace of the (possibly buggy)
execution~\footnote{If no witness could be found we simply return a trace
  of the last attempt.}.
%
We initially display a big-step reduction, \ie the source term followed
by the stuck term (or the final value).
%
The user can then progressively fill in the hidden steps of the
computation by selecting a visible term and choosing one of the
applicable traversal strategies -- described in
\S~\ref{sec:traversing-graph} -- to insert another term into the
visualization.

A sample interaction of the student program in
Figure~\ref{fig:palindrome} can be seen in
Figure~\ref{fig:nanomaly-palindrome}.
%
The initial state of the visualization tells us that the witness is
@palindrome []@, and that after some number of steps -- the thick arrow
denotes a multi-step transition -- we try to compare a function to a
list, which is not allowed in \ocaml.
%
Based on our semantics for \lang one would expect the final state to be
the \stuck term, but in practice we simply halt execution at the point
where an expression would transition to \stuck, as this final state is
more informative.
%
Note also that the equality test is colored black while the contextual
if-expression is faded out, this tells the user that the program got
stuck while trying to evaluate the sub-expression.

Upon seeing the stuck term, one might wonder where the @function@
came from.
%
To investigate we select the stuck term and click the ``jump backward''
button to search backwards from the stuck term for the most recent
function call, which brings us to @listReverse [] = w@, in the same
context as before.
%
Uncontent with the explanation so far, we ``step forward'' twice from
the @listReverse []@ term, bringing us to @helper [] = w@.
%
At this point it is clear that the @helper@ function is not defined
correctly, we have supplied it with the single argument we expected and
yet it still returned a @function@.

The problem is that the @function@ keyword in \ocaml defines an
anonymous function that takes a single argument and immediately does a
case-analysis without giving the argument a name.
%
The solution is to replace @function@ with an explicit @match xs with@
-- naming the value we wish to case-analyse.
%
After applying our fix, \nanomaly -- and more importantly \ocaml --
decide that @listReverse@ is safe to run.
\ES{these last few paragraphs probably belong in the overview}
%
\begin{figure}[t]
\centering
\begin{code}
  let listReverse l =
    let rec helper xs = function
      | [] -> xs
      | hd::tl -> helper (hd :: xs) tl
    in helper []

  let palindrome w =
    if listReverse w = w
    then true
    else false
\end{code}
\caption{Ill-typed \texttt{palindrome} function. The error lies in the
  use of the \texttt{function} construct which introduces an extra,
  undesired lambda abstraction, but manifests itself in the equality
  test on line XX. The fix is to replace \texttt{function} with
  \texttt{match xs with}. \ES{I think we should use this example in the
    intro/overview, leaving it here for now until they exist..}}
\label{fig:palindrome}
\end{figure}
%
\begin{figure}[t]
% \centering
Initial state: \\
\includegraphics[width=\linewidth]{palindrome} \\
After ``jumping backward'' from the stuck term: \\
\includegraphics[width=\linewidth]{palindrome2} \\
After ``stepping forward'' twice from @listReverse []@: \\
\includegraphics[width=\linewidth]{palindrome3} \\
\caption{Sample interaction}
\label{fig:nanomaly-palindrome}
\end{figure}

% !TEX root = main.tex


% \section{Evaluation: \nanomaly}                % 1 page
% \section{Evaluation: \nanomaly}
\label{sec:evaluation-nanomaly}
We have implemented our interactive semantics for the pure subset of
\ocaml with a web-based visualization of the generated traces in a tool
called \nanomaly.
%
It takes as input an \ocaml program and a function to check, searches
for a witness as described in \S~\ref{sec:searching-witness}, and
produces as output an interactive trace of the (possibly buggy)
execution~\footnote{If no witness could be found we simply return a trace
  of the last attempt.}.

We have not really (\ES{huh?}) evaluated our prototype of \nanomaly in two phases:
%
\begin{enumerate}
\item We investigate the \emph{complexity} of the generated traces by various size metrics.
\item We investigate the \emph{usefulness} of the generated traces compared to the standard compiler errors by performing an A/B study.
\end{enumerate}

\subsection{Trace Complexity}
\label{sec:trace-complexity}
For each of the ill-typed programs from \S~\ref{sec:eval-witness} for
which we could find a witness, we measure the complexity of the
generated trace according to the following metrics:
%
\begin{enumerate}
\item The number of single-step reductions along the path from the
  witness to the stuck term. This can be thought of as a worst-case
  complexity, \ie ``How big is the fully-expanded trace?''
\item The number forward (or backward) jumps that can be taken along the
  path from the witness to the stuck term. These abstract many details
  of the computation that are often uninteresting or irrelevant to the
  explanation.
\item \ES{others?}
\end{enumerate}
%
\begin{figure*}[t]
\centering
% \includegraphics[width=\linewidth]{trace_size.pdf}
\begin{minipage}{0.49\linewidth}
\includegraphics[width=\linewidth]{trace_size_step.pdf}
\end{minipage}
\begin{minipage}{0.49\linewidth}
\includegraphics[width=\linewidth]{trace_size_jump.pdf}
\end{minipage}
\caption{Complexity of the generated traces.}
\label{fig:results-complexity}
\end{figure*}
%
The results of the experiment are summarized in
Figure~\ref{fig:results-complexity}.
%
The average number of single-step reductions is XX with a minimum of XX
and a maximum of XX. The average number of jumps is YY with a minimum of
YY and a maximium of YY.

\subsection{User Study}
\label{sec:user-study}
We also performed an A/B study at the University of Virginia (IRB
\#XXXX). 
%
Participants were recruited from the graduate-level Programming
Languages course CSYYY, with the possibility of winning a \$50 gift
certificate upon completion.
%
We selected a random sample of nine ill-typed programs from our
benchmark dataset and from submissions to our public-facing demo of
\nanomaly.
%
For each program, we presented participants with the program and
alternatingly, either \ocaml's type error message or our interactive
visualization of a preset trace.
%
In order to alleviate any potential bias in problem selection, we also
alternated whether participants began with \ocaml or \nanomaly.
%
Participants were asked to identify the \emph{source} of the error,
\emph{explain} the error, and finally \emph{fix} it.

We collected the time taken for each problem and recruited a colleague
to grade the submissions, hiding whether a given solution came from the
\ocaml or \nanomaly group.

\paragraph{Results}
\label{sec:study-results}
\ES{TODO!}

\begin{figure}
\judgementHead{Evaluation}{\stepsto{e, \su}{e, \su}}
$$
\inference[\recontext]{\stepsto{e, \su}{e_1, \su_1}}{\stepsto{C[e], \su}{C[e_1], \su_1}}
$$

$$
\inference[\restuck]{}{\stepsto{C[\stuck], \su}{\stuck, \su}}
$$

$$
\inference[\replusgood]
  {n_1, \su_2 = \force{v_1}{\tnat} \\
   n_2, \su_3 = \force{v_2}{\tnat} \\ 
   n = \eplus{n_1}{n_2}}
  {\stepsto{\eplus{v_1}{v_2}, \su_1}{n, \su_1\su_2\su_3}}
$$

$$
\inference[\replusbadone]
  {\stuck, \su_2 = \force{v_1}{\tnat}}
  {\stepsto{\eplus{v_1}{v_2}, \su_1}{\stuck, \su_1\su_2}}
$$

$$
\inference[\replusbadtwo]
  {\stuck, \su_2 = \force{v_2}{\tnat}}
  {\stepsto{\eplus{v_1}{v_2}, \su_1}{\stuck, \su_1\su_2}}
$$

$$
\inference[\reifgoodone]
  {\etrue, \su_2 = \force{v}{\tbool}}
  {\stepsto{\eif{v}{e_1}{e_2}, \su_1}{e_1, \su_1\su_2}}
$$

$$
\inference[\reifgoodtwo]
  {\efalse, \su_2 = \force{v}{\tbool}}
  {\stepsto{\eif{v}{e_1}{e_2}, \su_1}{e_2, \su_1\su_2}}
$$

$$
\inference[\reifbad]
  {\stuck, \su_2 = \force{v}{\tbool}}
  {\stepsto{\eif{v}{e_1}{e_2}, \su_1}{\stuck, \su_1\su_2}}
$$
\caption{Evaluation relation}
\label{fig:evaluation}
\end{figure}


% \section{Related Work}              % 1 pages
\section{Related Work}
\label{sec:related-work}

\subsection{Type Error Localization}
\label{sec:type-error-local}
It is well known that unification-based type inference procedures can
produce poor error messages, and in particular, can misidentify the
\emph{source} of the type error.
%


\begin{itemize}
\item Lerner \etal~\cite{lerner_searching_2007} attempt to localize the
  type error and suggest a fix by replacing expressions (or removing
  them entirely) with alternatives based on the surrounding program
  context.
\item Chen and Erwig~\cite{chen_counter-factual_2014} use a variational
  type system represent the possibility that the inferred type of an
  expression may be incorrect, because the expression may be the source
  of the error. It then attempts to deduce the error source by searching
  for an expression whose type can be changed such that type inference
  would succeed. In contrast to Seminal, which searches for changes at
  the value-level, by searching at the type level the search is complete
  due the finite universe of types in a program.
\item Neubauer and Thiemann~\cite{neubauer_discriminative_2003} present
  a decidable type system based on discriminative sum types, in which
  all terms are typeable and type derivations contain all type errors in
  a program. They then use the typing derivation to slice out the parts
  of the expression related to each error.
\item Zhang and Myers~\cite{zhang_toward_2014} present an algorithm for
  identifying the most likely culprit in a system of unsatisfiable
  constraints (\eg type equalities), based on Bayesian reasoning.
\item Pavlinovic \etal~\cite{pavlinovic_finding_2014} translate the
  error localization problem to a MaxSMT optimization problem, using
  compiler-provided weights to rank the possible sources.
% \item \cite{chen_error-tolerant_2012}
% \item \cite{okeefe_type_1992}
% \item \cite{gomard_partial_1990}
% \item \cite{thatte_type_1988}
\end{itemize}

In contrast to these approaches, we do not attempt to localize or fix
the type error. Instead we try to explain it to the novice user using a
dynamic witness that demonstrates how their program is not just
ill-typed but truly wrong. In addition, allowing users to run their
program (even knowing that it is wrong) enables experimentation and the
use of debuggers to step through the program and investigate its
evolution.

\subsection{Testing}
\label{sec:testing}
\nanomaly is at its heart a test generator, and builds on a rich line of
work.
%
Our use of holes to represent unknown values is inspired by the work of
Runciman, Naylor, and Lindblad~\cite{runciman_smallcheck_2008,naylor_finding_2007,lindblad_property_2007},
%
who use lazy evaluation to drastically reduce the search space for
exhaustive test generation, by grouping together equivalent inputs by
the set of values they force. An exhaustive search is complete (up to
the depth bound), if a witness exists it will be found, but due to the
exponential blowup in the search space the depth bound can be quite
limited without advanced grouping and filtering techniques.
%
Our search is not exhaustive; instead we use random generation to fill
in holes on demand.
%
Random test generation~\cite{claessen_quickcheck:_2000,csallner_jcrasher:_2004,pacheco_feedback-directed_2007}
%
is by its nature incomplete, but is able to check larger inputs than
exhaustive testing as a result.

Instead of enumerating values, which may trigger the same path through
the program, one might enumerate paths. 
%
Dynamic-symbolic execution~\cite{godefroid_dart:_2005,cadar_klee:_2008,tillmann_pex_2008}
%
enables this by symbolic execution (to track which path a given input
vector triggers) with concrete execution (to ensure test failures are
not spurious). The system collects a path condition during execution,
which tracks symbolically what conditions must be met to trigger the
current path. Upon successfully completing a test run, it negates the
path condition and queries a solver for another set of inputs that
satisfy the negated path condition, \ie inputs that will not trigger the
same path. Thus, dynamic-symbolic execution can prune the search space
much faster than techniques based on enumerating values, but is limited
by the expressiveness of the underlying solver. Our operational
semantics is amenable to dynamic-symbolic execution, one would just need
to collect the path condition and replace our implementation of \gensym
by a call to the solver. We chose to use lazy, random generation instead
because it does not incur the overhead of an external solver, and
produces high coverage for our domain of novice programs.

% \begin{itemize}
% \item \cite{claessen_quickcheck:_2000}
% \item 
% \item \cite{godefroid_dart:_2005}
% \item \cite{cadar_klee:_2008}
% \item \cite{tillmann_pex_2008}
% \end{itemize}

\subsection{Misc}
\begin{itemize}
\item \cite{vytiniotis_equality_2012} extend the Haskell compiler GHC to
  support compiling ill-typed programs, but their intent is rather
  different from ours. Their goal was to allow programmers to
  incrementally test refactorings, which often cause type errors in
  distant pieces of code. They replace any expression that fails to
  type check with a \emph{runtime} error, but do not perform
  any type checking at runtime.
\item \cite{perera_functional_2012}
\end{itemize}


%%% Local Variables: 
%%% mode: latex
%%% TeX-master: "main"
%%% End: 


% \section{Conclusion}                % 1 page

{
\bibliographystyle{abbrvnat}
\bibliography{main}
}

\end{document}
